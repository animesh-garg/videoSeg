\section{Problem Formulation}
\label{sec:ProbForm}
\subsubsection*{Notations and Variables}
\begin{itemize}
\item We denote the video volume by $I$. A pixel in $I$ is indexed by
its location in space as well as time and is denoted by $I_{ijt}$
\item We wish to recover a complete segmentation of the video into foreground
and background. This labelling is captured by the variable $X$ where
$X_{ijt}\in\{0,1\}$
\item The time continuity between frames in a video implies that any pixel
in a given frame corresponds to some pixel in the next frame. We capture
this notion by a weak correspondence between a pixel and its neighbors
in the next frame. The correspondence weights for a pixel are denoted
by $W_{ijt}^{ab}$ ($a,b\in\{-h,..,h\}$) i.e we define a correspondence
weight variable between each pixel and the $(2h+1)X(2h+1)$ grid surrounding
it in the next frame.
\item By $N_{s}(i,j,t)$, we denote the indices of the pixels in the spatial
neighborhood of the pixel $(i,j,t)$
\item We also define variables $U,V,\overline{U},\overline{V}$ which capture the average motion direction of a pixel between consecutive frames in X, Y directions respectively.
We also denote the average direction of motion of the neighborhood
of a pixel by pseudo-variables $(\overline{U}_{ijt},\overline{V}_{ijt}).$ These
variables are defined in terms of the previously defined variables
as follows -
\begin{equation}
U_{ijt}=\underset{a,b\in\{-h,..,h\}}{\sum}aW_{ijt}^{ab}
\end{equation}
\begin{equation}
V_{ijt}=\underset{a,b\in\{-h,..,h\}}{\sum}bW_{ijt}^{ab}
\end{equation}
\begin{equation}
(\overline{U}_{ijt},\overline{V}_{ijt})=\frac{1}{|N_{s}(i,j,t)|}\underset{Y\in N_{s(i,j,t)}}{\sum}(U_{Y},V_{Y})
\end{equation}

\item Note that given $U$ and  $V$, we can recover the location that a given pixel gets mapped to in the next frame. Given this location, we can find interpolation weights for the surrounding pixels in the next frame and obtain a feasible $W$. Therefore, we can obtain $W$ given $U,V$ (and vice-versa as shown above). In the subsequent sections, we will define objectives and constraints in terms of $U,V,W$ but not all of them will be 'real' variables. It should be clear from the context which variables are being optimized over and which ones being used for notational convenience.

\end{itemize}

\subsubsection*{Objective}

\begin{equation}
\underset{X,W}{\min}\enskip\lambda_{1}A(X,I)+\lambda_{2}S(X)+\lambda_{3}T(X,W)
\end{equation}
\[
+\lambda_{4}F(W,I) +\lambda_{5}C(W)+\lambda_{6}M(W)
\]


\begin{center}
subject to $W\geq0,\forall(i,j,t)X_{ijt}\in\{0,1\},\underset{a,b}{\sum}W_{ijt}^{ab}=1$
and $\forall t\underset{i,j}{|\sum}X_{ijt}-\underset{i,j}{\sum}X_{ij(t+1)}|\leq\sigma\underset{i,j}{\sum}X_{ijt}$
\par\end{center}

\noindent The objective function comprises of various penatly terms
which are explained below. The last constraint specifies that the
number of foreground pixels in do not change rapidly between consecutive
frames.


\subsubsection*{Appearance Model $A(X,I)$}

Given the initial user labelled segmentation $X'$, we can form a
foreground model and a corresponding penalty function $f_{I,X'}$
for a pixel's label given its value. We then define the unary potential
as follows -

\begin{equation}
A(X,I)=\underset{i,j,t}{\sum}f_{I,X'}(X_{ijt},I_{ijt})
\end{equation}



\subsubsection*{Spatial Labelling Coherence $S(X)$}

We want to drive the system towards a labelling where neighbouring
pixels have similar labels. The spatial labelling coherence term defined
below encapsulates this.

\begin{equation}
S(X)=\underset{i,j,t}{\sum}\quad\underset{Y\in N_{s}(i,j,t)}{\sum}|X_{ijt}-X_{Y}|
\end{equation}



\subsubsection*{Temporal Labelling Coherence $T(X,W)$}

For a given pixel, the corresponding pixel in the next frame should
also have the same label. We formalize this notion using the penalty
function below.

\begin{equation}
T(X,W)=\underset{i,j,t}{\sum}\quad\underset{a,b\in\{-h,..,h\}}{\sum}W_{ijt}^{ab}|X_{ijt}-X_{i+a,j+b,t+1}|
\end{equation}



\subsubsection*{Flow Similarity $F(W,I)$}

For each pixel, the corresponding pixel in the next frame should be
similar. This is enforced by the flow similarity defined below.

\begin{equation}
F(X,I)=\underset{i,j,t}{\sum}\quad\underset{a,b\in\{-h,..,h\}}{\sum}W_{ijt}^{ab}|I_{ijt}-I_{i+a,j+b,t+1}|
\end{equation}



\subsubsection*{Flow Continuity $C(W)$}

The direction of movement of pixels is continuous over a small spatial
neighbourhood. We therefore penalize rapid variations in flow as follows-

\begin{equation}
C(W)=\underset{i,j,t}{\sum}|U_{ijt}-\overline{U}_{ijt}|+|V_{ijt}-\overline{V}_{ijt}|
\end{equation}



\subsubsection*{Momentum Continuity $M(W)$}

It also needs to be enforced that the velocity of a pixel and its
corresponding pixel in the next frame do not vary rapidly. This is
ensured by the momentum continuity terms defined below

\begin{equation}
M(W)=\underset{i,j,t}{\sum}\quad\underset{a,b\in\{-h,..,h\}}{\sum}W_{ijt}^{ab}(|a-\overline{U}_{i+a,j+b,t+1}|+
\end{equation}
\[|b-\overline{V}_{i+a,j+b,t+1}|)
\]


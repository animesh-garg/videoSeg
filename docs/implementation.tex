\section{Implementation Details}

Note that the reformulation mentioned above does not necessitate any
change in the proposed optimization algorithm as we still need to
follow the block-gradient descent method (since even the reformulation
does not make the objective jointly convex in $X,W$ and $X$ is still
required to be discreet). In our implementation, we experiment with both the relaxed formulation as well as the original formulation (with momentum penalty modified to be convex). In this section, we describe the finer implementation details of our algorithm.


\subsection{Initialization}

We have shown earlier that the objective function is not jointly convex
over $X,W$. So, the solution that the algorithm converges to is highly
dependent on the initialization. Therefore, we need to determine a
reasonable initialization for the weight variables. We experiment
with two standard optical flow algorithms \cite{HornSchunk}, \cite{LukasKanade} to
initialize the flow variables for each video frame (and thus the weight
variables). The implementation is as follows -

\begin{algorithm}[H]
\framebox{\begin{minipage}[t]{1\columnwidth}%
function $generatePriors(I)$
\begin{itemize}
\item for $t=1:T-1$

\begin{itemize}
\item $(U_{t},V_{t})=opticalFlow(I_{t},I_{t+1})$
\end{itemize}
\item $W\leftarrow UVtoWeights(U,V)$
\item return $W$\end{itemize}
%
\end{minipage}}\caption{$generatePriors(I)$}
\end{algorithm}



\subsection{Unary Appearance Cost}

To determine the unary costs, we learn a foreground model based on
the intitial frame segmentation. We train a random forest classifier
using for patches using the patches around foreground pixels in the
intial frame as positive examples and the patches around other pixels
as negative examples. We then predict the probability of a pixel in
a given frame being a foreground pixel by classifying the patch surrounding
it using the trained classifier. The foreground label cost for the
given pixel is stored as $1-classifierProbability(I_{ijt})$.


\subsection{Finding flows given label assignments}

To find the optimal flows given the label assignments, we need only consider the $T, F, C$, and $M$ costs from the original objective because the other costs do not penalize flow. The only constraint is a lower and upper bound on the flows, to ensure that they lie within some temporal winodw. The function now consists of a number of different L1 penalties of linear functions of the flow, which can be converted to an LP using the epigraph form. We introduce an auxiliary variable for each absolute value appearing in the objective, yielding 8 auxiliary variables for each pixel in the video (supposing 3 color channels). For example, the penalty $\sum_{i j t} |U_{i j t} - \bar{U}_{i j t} |$ is converted to the penalty $\sum_{i j t} R_{i j t} : R_{i j t} \geq \pm (U_{i j t} - \bar{U}_{i j t})$. 

Once the problem is in epigraph form, we put the problem in standard LP form by placing the relevant coefficients for the penalties and linear inequalities into vectors. One difficulty is making this conversion is respecting the image boundaries, since the spatial and temporal windows will exceed the image boundaries. We handle this case by shrinking the window size to fit only the portion of the window that lies within image bounds when placing the linear coefficients into the cost and inequality matrices.


\subsection{Propagating labels}
Given the initial segmentation mask, the flows for the complete sequence, we run the
$propagateLabels~$ module to get the segmentation mask through out the video 
sequence. 
The sub-problem we consider at this stage considers: (a) Appearance Model Eq
\eqref{eq: appModel} , (b) Spatial Labeling Constraint Eq\eqref{eq: SLC}, (c) Track 
Shrinkage Constraint Eq\eqref{eq: shrinkage} and (d) Temporal Labeling Constraint Eq
\eqref{eq: TLC}.
The terms in the objective for the corresponding block of variables are included. 
Each term of the form: $|x|$ is rewritten as a set of 2 constraints:$x \leq t$ and $-x \leq t$, where $t$ is an auxiliary variable. 

The reduced the problem to a \textit{Mixed Integer Problem} for which we use a 
Matlab based interface to CPLEX solver. We found that the performance of MOSEK 
solver on MIPs was worse. Noticeably the problem structure uses problem specific information only in the constraints which involve using $W's$ (flows/weights). Problem
structure is constant for all other sets of constraints, which allows us to re-use the 
same problem model in subsequent iterations. This model caching results in
substantial speed up in problem generation time, which we found empirically 
dominates the problem solution time. 

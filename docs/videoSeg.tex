%%%%%%%%%%%%%%%%%%%%%%%%%%%%%%%%%%%%%%%%%%%%%%%%%%%%%%%%%%%%%%%%%%
%%%%%%%% ICML 2014 EXAMPLE LATEX SUBMISSION FILE %%%%%%%%%%%%%%%%%
%%%%%%%%%%%%%%%%%%%%%%%%%%%%%%%%%%%%%%%%%%%%%%%%%%%%%%%%%%%%%%%%%%

% Use the following line _only_ if you're still using LaTeX 2.09.
%\documentstyle[icml2014,epsf,natbib]{article}
% If you rely on Latex2e packages, like most moden people use this:
\documentclass{article}

% use Times
\usepackage{times}
% For figures
\usepackage{graphicx} % more modern
%\usepackage{epsfig} % less modern
\usepackage{subfigure} 

% For citations
\usepackage{natbib}

% For algorithms
\usepackage{algorithm}
\usepackage{algorithmic}

% As of 2011, we use the hyperref package to produce hyperlinks in the
% resulting PDF.  If this breaks your system, please commend out the
% following usepackage line and replace \usepackage{icml2014} with
% \usepackage[nohyperref]{icml2014} above.
\usepackage{hyperref}

% Packages hyperref and algorithmic misbehave sometimes.  We can fix
% this with the following command.
\newcommand{\theHalgorithm}{\arabic{algorithm}}

%USER defined Packages
\usepackage{amsmath}


% Employ the following version of the ``usepackage'' statement for
% submitting the draft version of the paper for review.  This will set
% the note in the first column to ``Under review.  Do not distribute.''
%\usepackage{icml2014} 
% Employ this version of the ``usepackage'' statement after the paper has
% been accepted, when creating the final version.  This will set the
% note in the first column to ``Proceedings of the...''
\usepackage[accepted]{icml2014}


% The \icmltitle you define below is probably too long as a header.
% Therefore, a short form for the running title is supplied here:
\icmltitlerunning{Video Segmentation as a Distributed Convex Problem}

\begin{document} 

\twocolumn[
\icmltitle{Video Segmentation as a Distributed Convex Optimization Problem\\
 using Primal Decomposition} 

% It is OKAY to include author information, even for blind
% submissions: the style file will automatically remove it for you
% unless you've provided the [accepted] option to the icml2014
% package.
\icmlauthor{Animesh Garg*}{animesh.garg@berkeley.edu}
\icmlauthor{Jeff Mahler*}{jmahler@berkeley.edu}
\icmlauthor{Shubham Tulsiani*}{shubhtuls@berkeley.edu}
\icmladdress{Department of EECS, UC Berkeley, CA 94720}

% You may provide any keywords that you 
% find helpful for describing your paper; these are used to populate 
% the "keywords" metadata in the PDF but will not be shown in the document
\icmlkeywords{Tracking, Segmentation, Video, Convex Optimization, Subgradient Learning}

\vskip 0.3in
]

\begin{abstract} 
Getting exact video segmentations for tracking and recognition is a challenging problem. 
A majority of existing methodstrack but provide a bounding box rather than a an exact foreground
mask for the object. For real worl applications of perception, like robotics, the silhoutte of the object
perhaps even pose need to be known for hope of success in manipulation tasks.

We propose a method in this study which formulated the problem of video segmentation
as a Markov random field. However solving such a large graph to global optimality may 
be computationally expensive. Hence we propose a distributed method using Primal 
decomposition. 

\end{abstract} 


\section{Introduction}

The problem of video segmentation is of interest for many areas. 
\cite{Komodakis2007a, Komodakis2011a} and \cite{Tsai2010}
have looked at the problem of modelling the MRF in terms of energies. 
The solution strategy they use is Dual decomposition but without integer programming.

Our study explores the use of state-of-the-art integer program solvers. Modelling integers
allows us to capture more rich features in video which are usually not directly put in current 
models. 
From a given video sequence, and user intialized object(s)
of interest, the aim is to track the region(s) of interest through
the subsequent image frames in the video. Majority of other
methods which address the problem provide locally optimal
solutions. Such an approach though successful in some applications
requires a substantial amount of human intervention
at several points in the solution such as in cases of occlusion,
change in pose, shape and color and in extreme cases object
(or a part of object) egresses the frame and re-enters later.
\section{Problem Formulation}
\label{sec:ProbForm}
\subsubsection*{Notations and Variables}
\begin{itemize}
\item We denote the video volume by $I$. A pixel in $I$ is indexed by
its location in space as well as time and is denoted by $I_{ijt}$
\item We wish to recover a complete segmentation of the video into foreground
and background. This labelling is captured by the variable $X$ where
$X_{ijt}\in\{0,1\}$
\item The time continuity between frames in a video implies that any pixel
in a given frame corresponds to some pixel in the next frame. We capture
this notion by a weak correspondence between a pixel and its neighbors
in the next frame. The correspondence weights for a pixel are denoted
by $W_{ijt}^{ab}$ ($a,b\in\{-h,..,h\}$) i.e we define a correspondence
weight variable between each pixel and the $(2h+1)X(2h+1)$ grid surrounding
it in the next frame.
\item By $N_{s}(i,j,t)$, we denote the indices of the pixels in the spatial
neighborhood of the pixel $(i,j,t)$
\item We define pseudo-variables $U,V,\overline{U},\overline{V}$ for notational
convenience. The variables $(U,V)$ capture the average motion direction
of a pixel between consecutive frames in X, Y directions respectively.
We also denote the average direction of motion of the neighborhood
of a pixel by $(\overline{U}_{ijt},\overline{V}_{ijt}).$ These pseudo
variables are defined in terms of the previously defined variables
as follows -
\begin{equation}
U_{ijt}=\underset{a,b\in\{-h,..,h\}}{\sum}aW_{ijt}^{ab}
\end{equation}
\begin{equation}
V_{ijt}=\underset{a,b\in\{-h,..,h\}}{\sum}bW_{ijt}^{ab}
\end{equation}
\begin{equation}
(\overline{U}_{ijt},\overline{V}_{ijt})=\frac{1}{|N_{s}(i,j,t)|}\underset{Y\in N_{s(i,j,t)}}{\sum}(U_{Y},V_{Y})
\end{equation}

\end{itemize}

\subsubsection*{Objective}

\begin{equation}
\underset{X,W}{\min}\enskip\lambda_{1}A(X,I)+\lambda_{2}S(X)+\lambda_{3}T(X,W)
\end{equation}
\[
+\lambda_{4}F(W,I) +\lambda_{5}C(W)+\lambda_{6}M(W)
\]


\begin{center}
subject to $W\geq0,\forall(i,j,t)X_{ijt}\in\{0,1\},\underset{a,b}{\sum}W_{ijt}^{ab}=1$
and $\forall t\underset{i,j}{|\sum}X_{ijt}-\underset{i,j}{\sum}X_{ij(t+1)}|\leq\sigma\underset{i,j}{\sum}X_{ijt}$
\par\end{center}

\noindent The objective function comprises of various penatly terms
which are explained below. The last constraint specifies that the
number of foreground pixels in do not change rapidly between consecutive
frames.


\subsubsection*{Appearance Model $A(X,I)$}

Given the initial user labelled segmentation $X'$, we can form a
foreground model and a corresponding penalty function $f_{I,X'}$
for a pixel's label given its value. We then define the unary potential
as follows -

\begin{equation}
A(X,I)=\underset{i,j,t}{\sum}f_{I,X'}(X_{ijt},I_{ijt})
\end{equation}



\subsubsection*{Spatial Labelling Coherence $S(X)$}

We want to drive the system towards a labelling where neighbouring
pixels have similar labels. The spatial labelling coherence term defined
below encapsulates this.

\begin{equation}
S(X)=\underset{i,j,t}{\sum}\quad\underset{Y\in N_{s}(i,j,t)}{\sum}|X_{ijt}-X_{Y}|
\end{equation}



\subsubsection*{Temporal Labelling Coherence $T(X,W)$}

For a given pixel, the corresponding pixel in the next frame should
also have the same label. We formalize this notion using the penalty
function below.

\begin{equation}
T(X,W)=\underset{i,j,t}{\sum}\quad\underset{a,b\in\{-h,..,h\}}{\sum}W_{ijt}^{ab}|X_{ijt}-X_{i+a,j+b,t+1}|
\end{equation}



\subsubsection*{Flow Similarity $F(W,I)$}

For each pixel, the corresponding pixel in the next frame should be
similar. This is enforced by the flow similarity defined below.

\begin{equation}
F(X,I)=\underset{i,j,t}{\sum}\quad\underset{a,b\in\{-h,..,h\}}{\sum}W_{ijt}^{ab}|I_{ijt}-I_{i+a,j+b,t+1}|
\end{equation}



\subsubsection*{Flow Continuity $C(W)$}

The direction of movement of pixels is continuous over a small spatial
neighbourhood. We therefore penalize rapid variations in flow as follows-

\begin{equation}
C(W)=\underset{i,j,t}{\sum}|U_{ijt}-\overline{U}_{ijt}|+|V_{ijt}-\overline{V}_{ijt}|
\end{equation}



\subsubsection*{Momentum Continuity $M(W)$}

It also needs to be enforced that the velocity of a pixel and its
corresponding pixel in the next frame do not vary rapidly. This is
ensured by the momentum continuity terms defined below

\begin{equation}
M(W)=\underset{i,j,t}{\sum}\quad\underset{a,b\in\{-h,..,h\}}{\sum}W_{ijt}^{ab}(|a-\overline{U}_{i+a,j+b,t+1}|+
\end{equation}
\[|b-\overline{V}_{i+a,j+b,t+1}|)
\]



\section{Experiments and Metrics}
\label{sec:Expt}
As presented in the model in Section~\ref{sec:ProbForm} we have a complete optimization
model with several integer variables for foreground-background labels.

We will test the performance of our solution on the 
Berkeley Motion Segmentation Dataset as provided by \cite{brox2010object}.
The dataset has 26 video sequences with pixel-accurate segmentation annotation of moving objects. A total of 189 frames are annotated.

We will evaluate results from our approach and compare the performance with that of \cite{Felzenszwalb2010b}, \cite{Komodakis2011a}
and \cite{brox2010object} on this dataset.

Furthermore multiple decoupling strategies will implementation and compared,
like decoupling time frames v/s decoupling in space. Finally a dual decomposition
method with also be explored and compared qualitatively with \cite{Komodakis2007a}.

We plan on completing the implementation in MATLAB with the use of CVX and CPLEX
optimization libraries.



% Acknowledgements should only appear in the accepted version. 
%\section*{Acknowledgments} 
% 
%\textbf{Do not} include acknowledgements in the initial version of
%the paper submitted for blind review.
%
%If a paper is accepted, the final camera-ready version can (and
%probably should) include acknowledgements. In this case, please
%place such acknowledgements in an unnumbered section at the
%end of the paper. Typically, this will include thanks to reviewers
%who gave useful comments, to colleagues who contributed to the ideas, 
%and to funding agencies and corporate sponsors that provided financial 
%support.  


% In the unusual situation where you want a paper to appear in the
% references without citing it in the main text, use \nocite
%\nocite{langley00}

\bibliography{videoSeg}
\bibliographystyle{icml2014}

\end{document} 

